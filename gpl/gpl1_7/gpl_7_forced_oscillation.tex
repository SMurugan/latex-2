\documentclass[a4paper]{article}
\usepackage{array, xcolor, lipsum, bibentry,fancyhdr}
\usepackage[scale=0.75,twoside,bindingoffset=5mm]{geometry}
\usepackage[onehalfspacing]{setspace}
\usepackage{amsmath, kotex, graphicx, amsthm}
\usepackage{multirow}

\newtheorem{theorem}{Theorem}%[section]
\newtheorem{corollary}{Corollary}[theorem]
\newtheorem{lemma}[theorem]{Lemma}
\newtheorem{remark}{Remark}

\pagestyle{fancy}
\lhead{}
\chead{}
\rhead{\thepage}
\renewcommand{\headrulewidth}{0.4pt}
%\renewcommand{\footrulewidth}{0.4pt}

\newcolumntype{P}[1]{>{\centering\arraybackslash}p{#1}}
\newcolumntype{M}[1]{>{\centering\arraybackslash}m{#1}}
\usepackage[utf8]{inputenc}

\author{16-046 백승재, 16-087 이현동}

\title{Lab 07. Forced Oscillation}
\begin{document}
\maketitle
%\tableofcontents

\begin{abstract}
Hello, world!
\end{abstract}

\section{Introduction}
This experiment is about simple, damped, and forced harmonic motion. In the spring-mass system, we can find out the change of velocity and analyze the motion of the mass.
\section{Theory}
% ###################################################
\subsection{Simple Harmonic Motion}
Simple Harmonic Motion (SHM) is given by
\begin{equation*}
    \frac{d^2x}{dt^2} + \frac{k}{m}x = 0
\end{equation*}
Solving this equation about $x$, we can get the solution $x(t)$:
\begin{equation*}
    x(t) = A \sin (\omega_0t + \theta_0) \qquad
    \omega_0 = \sqrt{\frac{k}{m}}
\end{equation*}
% ############################################################
\subsection{Damped Harmonic Motion}
Damped harmonic motion is different from the equation of SHM since it includes another term,
\begin{equation*}
    \frac{d^2x}{dt^2} + \frac{b}{m}\frac{dx}{dt} +\frac{k}{m}x = 0
\end{equation*}
which leads to the other solution of $x$:
\begin{equation*}
    x(t) = x_0e^{-\omega_{\gamma}t}\sin(\omega^\prime t)
\end{equation*}
where
\begin{equation*}
    \omega_\gamma = \frac{b}{2m} \qquad \omega^{\prime 2} = \omega_0^2 - \omega_\gamma^2
\end{equation*}
% ######################################################## 
\subsection{Damped Harmonic Motion}
Forced harmonic motion with driving force $F_d$ is expressed by the following equation.
\begin{equation*}
    m\frac{d^2x}{dt^2} + b\frac{dx}{dt} + kx = F_d(t)
\end{equation*}
By using the fact that $F_d(t) = F_d\cos(\omega_dt)$, this leads to the following equation.
\begin{equation*}
    \frac{d^2x}{dt^2} + \frac{b}{m}\frac{dx}{dt} + \frac{k}{m}x - \frac{F_d}{m}\cos(\omega_dt) = 0
\end{equation*}

% ################################################## 
\section{Procedure}

\section{Results and Analysis}

\section{Discussion}

\section{Conclusion}

\end{document}