\documentclass[12pt]{article}

\usepackage{sbc-template}

\usepackage{graphicx,url}
\usepackage{kotex}
\linespread{1.3}

\usepackage[utf8]{inputenc}  
% UTF-8 encoding is recommended by ShareLaTex
\usepackage{verbatim}
\usepackage{listings}
\usepackage{xcolor}

\definecolor{verde}{rgb}{0,0.5,0}

\title{Worksheet 2. Electric Charge Density}

\author{16-087 이현동}


\address{한국과학영재학교 일반물리학실험2 1분반
 \email{lhdsou@gmail.com}
}

\begin{document} 

\maketitle


\section{Charge Density Inside and on the Surface of a Cylinder}

\subsection{Result}
asdfas
\subsection{Discussions}

\begin{enumerate}
    \item{\textbf{원통 내부와 외부의 전하 밀도를 비교하라.}}
    \item[]{lol}
    \item{원통의 중심에서 가장자리까지 이동할 때 전하 밀도의 변화를 서술하라.}
        lol
    \item{무한히 긴 원통에 대하여 전하 밀도를 예측하라.}
\end{enumerate}

\section{Charge Density on a Plane near a Point Source}

\subsection{Result}
blash
\subsection{Discussion}

\begin{enumerate}
    \item{Proof Plane과 점전하 disk 사이의 거리와 전하 밀도 사이에 어떠한 관계가 있는가? 만약 있다면 어떤 관계인가?}
    
    \item{Conductive paper의 넓이가 무한할 때 전하 밀도는 어떻게 달라지는가?}
    
    \item[]{나도 몰라요}
    
\end{enumerate}

\end{document}
